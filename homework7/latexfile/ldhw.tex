% Homework template for Learning from Data
% by Xiangxiang Xu <xiangxiangxu.thu@gmail.com>
% LAST UPDATE: October 8, 2018
\documentclass[a4paper]{article}
\usepackage[T1]{fontenc}
\usepackage{amsmath, amssymb, amsthm}
% amsmath: equation*, amssymb: mathbb, amsthm: proof
\usepackage{moreenum}
\usepackage{mathtools}
\usepackage{url}
\usepackage{enumitem}
\usepackage{bm}
\usepackage{graphicx}
\usepackage{subcaption}
\usepackage{booktabs} % toprule
\usepackage[mathcal]{eucal}
\usepackage{dsfont}
\usepackage[numbered,framed]{matlab-prettifier}
\input{lddef}

\lstset{
  style              = Matlab-editor,
  captionpos         =b,
  basicstyle         = \mlttfamily,
  escapechar         = ",
  mlshowsectionrules = true,
}
\begin{document}
\courseheader



\newcounter{hwcnt}
\setcounter{hwcnt}{3} % set to the times of Homework

\begin{center}
  \underline{\bf Programming Homework 4}%\thehwcnt} \\
\end{center}
\begin{flushleft}
  TIAN Chenyu\hfill
  \today
\end{flushleft}
\hrule

\vspace{2em}
\setlist[enumerate,1]{label=\thehwcnt.\arabic*.}
\setlist[enumerate,2]{label=(\alph*)}
\setlist[enumerate,3]{label=\roman*.}
\setlist[enumerate,4]{label=\greek*)}

\flushleft
\rule{\textwidth}{1pt}
\begin{itemize}
\item {\bf Acknowledgments: \/} 
  This template takes some materials from course CSE 547/Stat 548 of Washington University: \small{\url{https://courses.cs.washington.edu/courses/cse547/17sp/index.html}}.
\item {\bf Collaborators: \/}
  I finish this homework by myself.
  % \begin{itemize}
  % \item 1.2 (b) was solved with the help from \underline{\hspace{3em}}.
  % \item Discussion with \underline{\hspace{3em}} helped me finishing 1.3.
  % \end{itemize}
\end{itemize}
\rule{\textwidth}{1pt}

\vspace{2em}

% You may use \texttt{enumerate} to generate answers for each question:

% 4.1
\begin{enumerate}
  \setlength{\itemsep}{4\parskip}
\item 
The matrix of learned reward matrix is shown below.

It is easy to see the most different part is in the (3, 0), where the reward is 2.
May because the part is in the corner and rarely visited leading to some bias.

% 4.2
\item My DQN is work, but it really takes more time than q_table because . But it is more accurate.
By the way, it is really hard to train for lacking of deep learning experience.
At each iteration of DQN, a mini-batch of states, actions, rewards, and next states are sampled from the replay memory as observations to train the Q-network, which approximates the action-value function.
In this discrete senario, I think Q learning is more efficient. But DQN is more suitable to
complex continuous tasks.


\end{enumerate}
  
  % \newpage
  
  % \appendix
  % \section{Source code}
  % \label{sec:a:code}
  % % \lstlistoflistings
  % Source code for plotting Figure \ref{fig:1} is shown as follows.
  % \lstinputlisting{matlabscript.m}
  
\end{document}
%%% Local Variables:
%%% mode: latex
%%% TeX-master: t
%%% End:
