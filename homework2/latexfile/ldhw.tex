% Homework template for Learning from Data
% by Xiangxiang Xu <xiangxiangxu.thu@gmail.com>
% LAST UPDATE: October 8, 2018
\documentclass[a4paper]{article}
\usepackage[T1]{fontenc}
\usepackage{amsmath, amssymb, amsthm}
% amsmath: equation*, amssymb: mathbb, amsthm: proof
\usepackage{moreenum}
\usepackage{mathtools}
\usepackage{url}
\usepackage{enumitem}
\usepackage{bm}
\usepackage{graphicx}
\usepackage{subcaption}
\usepackage{booktabs} % toprule
\usepackage[mathcal]{eucal}
\usepackage{dsfont}
\usepackage[numbered,framed]{matlab-prettifier}
%% Definitions for Learning from Data
%% UPDATE: October 8, 2018 by Xiangxiang 
\newcommand{\theterm}{Fall 2019}

\newcommand{\thecoursename}{
Tsinghua-Berkeley Shenzhen Institute\\
%\vspace*{0.1in}
\textsc{Learning from Data}
}

\newcommand{\courseheader}{
\vspace*{-1in}
\begin{center}
\thecoursename \\
\theterm
\vspace*{0.1in}
\hrule
\end{center}
}
\newcommand{\uc}{\underline{c}}    % c, vec
\newcommand{\uv}{\underline{v}}    % x, vec
\newcommand{\uw}{\underline{w}}    % w, vec
\newcommand{\ux}{\underline{x}}    % x, vec
\newcommand{\uy}{\underline{y}}    % y, vec
\newcommand{\uz}{\underline{z}}    % z, vec
\newcommand{\um}{\underline{m}}    % m, vec
\newcommand{\ut}{\underline{t}}    % t, vec
\newcommand{\bc}{\bm{c}}    % c, vec
\newcommand{\bv}{\bm{v}}    % x, vec
\newcommand{\bw}{\bm{w}}    % w, vec
\newcommand{\bx}{\bm{x}}    % x, vec
\newcommand{\by}{\bm{y}}    % y, vec
\newcommand{\bz}{\bm{z}}    % z, vec
% \newcommand{\bm}{\bm{m}}    % m, vec
\newcommand{\bt}{\bm{t}}    % t, vec

\newcommand{\balpha}{\bm{\alpha}}    % alpha, vec
\newcommand{\bxi}{\bm{\xi}}    % xi, vec


\newcommand{\rvx}{\mathsf{x}}    % x, r.v.
\newcommand{\rvy}{\mathsf{y}}    % y, r.v.
\newcommand{\rvz}{\mathsf{z}}    % z, r.v.
\newcommand{\rvw}{\mathsf{w}}    % w, r.v.
\newcommand{\rvv}{\mathsf{v}}    % v, r.v.
\newcommand{\rvm}{\mathsf{m}}    % m, r.v.
\newcommand{\rvt}{\mathsf{t}}    % t, r.v.
\newcommand{\rvH}{\mathsf{H}}    % H, r.v.
\newcommand{\urvx}{\underline{\mathsf{x}}}    % x, r.v. vec
\newcommand{\urvy}{\underline{\mathsf{y}}}    % y, r.v. vec
\newcommand{\urvz}{\underline{\mathsf{z}}}    % z, r.v. vec
\newcommand{\urvw}{\underline{\mathsf{w}}}    % w, r.v. vec
\newcommand{\urvt}{\underline{\mathsf{t}}}    % t, r.v. vec
\newcommand{\defeq}{\triangleq} %\coloneqq
\newcommand{\reals}{\mathbb{R}}
\newcommand{\T}{\mathrm{T}}    % transpose
\newcommand{\BLS}{\mathrm{BLS}}    % BLS
\newcommand{\LLS}{\mathrm{LLS}}    % LLS
\newcommand{\MVU}{\mathrm{MVU}}    % MVU

\DeclareMathOperator*{\maximize}{maximize}    % maximize
\DeclareMathOperator*{\minimize}{minimize}    % minimize
\newcommand{\st}{\mathrm{subject~to}}    % minimize


% \newcommand{\E}[1]{\mathbb{E}\left[{#1}\right]}
% \newcommand{\Prob}[1]{\mathbb{P}\left({#1}\right)}
\DeclareMathOperator*{\argmax}{arg\,max}
\DeclareMathOperator*{\argmin}{arg\,min}
\DeclareMathOperator*{\argsup}{arg\,sup}
\DeclareMathOperator*{\arginf}{arg\,inf}
\DeclareMathOperator{\Var}{Var}
\DeclareMathOperator{\Cov}{Cov}
\DeclareMathOperator{\MSE}{MSE}
\DeclareMathOperator{\1}{\mathds{1}}
\DeclareMathOperator{\E}{\mathbb{E}}
\DeclareMathOperator{\Prob}{\mathbb{P}}

\newcommand\independent{\protect\mathpalette{\protect\independenT}{\perp}}
\def\independenT#1#2{\mathrel{\rlap{$#1#2$}\mkern2mu{#1#2}}}


\lstset{
  style              = Matlab-editor,
  captionpos         =b,
  basicstyle         = \mlttfamily,
  escapechar         = ",
  mlshowsectionrules = true,
}
\begin{document}
\courseheader



\newcounter{hwcnt}
\setcounter{hwcnt}{1} % set to the times of Homework

\begin{center}
  \underline{\bf Homework \thehwcnt} \\
\end{center}
\begin{flushleft}
  TIAN Chenyu\hfill
  \today
\end{flushleft}
\hrule

\vspace{2em}
\setlist[enumerate,1]{label=\thehwcnt.\arabic*.}
\setlist[enumerate,2]{label=(\alph*)}
\setlist[enumerate,3]{label=\roman*.}
\setlist[enumerate,4]{label=\greek*)}

\flushleft
\rule{\textwidth}{1pt}
\begin{itemize}
\item {\bf Acknowledgments: \/} 
  This template takes some materials from course CSE 547/Stat 548 of Washington University: \small{\url{https://courses.cs.washington.edu/courses/cse547/17sp/index.html}}.
\item {\bf Collaborators: \/}
  I finish this homework by myself.
  % \begin{itemize}
  % \item 1.2 (b) was solved with the help from \underline{\hspace{3em}}.
  % \item Discussion with \underline{\hspace{3em}} helped me finishing 1.3.
  % \end{itemize}
\end{itemize}
\rule{\textwidth}{1pt}

\vspace{2em}

% You may use \texttt{enumerate} to generate answers for each question:

\begin{enumerate}
  \setlength{\itemsep}{3\parskip}
\item The log-likelihood of the softmax regression model can be writen as  
\begin{equation*}
    % \tag{P}
    \begin{aligned}
      l 
      &= \sum_{i = 1}^m \log \frac{\exp (\boldsymbol{\theta}^T_{y^{(i)}} \boldsymbol{x}^{(i)}+b_{y^{(i)} })}{\sum_{j = 1}^k \exp(\boldsymbol{\theta}^T_j \boldsymbol{x}^{(i)}+b_j)}\\
      &= \sum_{i = 1}^m [\boldsymbol{\theta}^T_{y^{(i)}} \boldsymbol{x}^{(i)}+b_{y^{(i)}} - \log (\sum_{j = 1}^k \exp(\boldsymbol{\theta}^T_j \boldsymbol{x}^{(i)}+b_j))]\\
    \end{aligned}
    \label{eq:svm:p}
  \end{equation*}

  \begin{enumerate}
  \item Evaluate the derivation of $b_l$:
    \begin{equation*}
      \frac{\partial l}{\partial b_l} = \sum_{i = 1}^m [\1 (y^{(i)}=l) - \frac{\exp(\boldsymbol{\theta}^T_l \boldsymbol{x}^{(i)}+b_l)}{\sum_{j = 1}^k \exp(\boldsymbol{\theta}^T_j \boldsymbol{x}^{(i)}+b_j)}]
    \end{equation*}
    \noindent The $\1 (y^{(i)}=l)$ function is defined as:
    \begin{equation*}
      \1(x=l)=
      \left\{
      \begin{aligned}
        1, &\quad\text{if}~ x = l,\\
        0, &\quad\text{if}~ x \neq l.
      \end{aligned}
      \right.
    \end{equation*}
   
  \item
  If we have set the biases to their optimal values, there exists $\frac{\partial l}{\partial b_l} = 0$.
  Based on (a):
  \begin{equation*}
    \sum_{i = 1}^m \1 (y^{(i)}=l) = \frac{\exp(\boldsymbol{\theta}^T_l \boldsymbol x^{(i)}+b_l)}{\sum_{j = 1}^k \exp(\boldsymbol{\theta}^T_j \boldsymbol x^{(i)}+b_j)}
  \end{equation*}
  Based on the definition of $\hat{P}_y (l)$:
  \begin{equation*}
    \begin{aligned}
    \hat{P}_y (l) 
    &= \frac{1}{m} \sum_{i = 1}^m \1 (y^{(i)}=l)\\
    &= \frac{1}{m} \sum_{i = 1}^m \frac{\exp(\boldsymbol{\theta}^T_l \boldsymbol x^{(i)}+b_l)}{\sum_{j = 1}^k \exp(\boldsymbol \theta^T_j \boldsymbol x^{(i)}+b_j)}\\
    &= \frac{1}{m} \sum_{i = 1}^m \sum_{\boldsymbol x \in \boldsymbol X} \frac{\exp(\boldsymbol \theta^T_l \boldsymbol x+b_l)}{\sum_{j = 1}^k \exp(\boldsymbol \theta^T_j \boldsymbol x^{(i)}+b_j)}\1 (\boldsymbol x^{(i)} = \boldsymbol x)\\
    &= \frac{1}{m} \sum_{i = 1}^m \sum_{\boldsymbol x \in \boldsymbol X} P_{(y \mid \boldsymbol x)} (l \mid \boldsymbol x) \1 (\boldsymbol x^{(i)} = \boldsymbol x)\\
    &= \sum_{\boldsymbol x \in \boldsymbol X} P_{(y \mid \boldsymbol x)} (l \mid \boldsymbol x) \frac{1}{m} \sum_{i = 1}^m \1 (\boldsymbol x^{(i)} = \boldsymbol x)\\
    &= \sum_{\boldsymbol x \in \boldsymbol X} P_{(y \mid \boldsymbol x)} (l \mid \boldsymbol x) \hat{P}_{\boldsymbol x} (\boldsymbol x) 
    \end{aligned}
  \end{equation*}
  \end{enumerate}  

\item
  \begin{enumerate}
  \item
    The MLE is to solve
    \begin{equation*}
      \maximize_{\mu} \prod_{i=1}^n P(x_i \mid \mu)
    \end{equation*}
    which is equal to:
    \begin{equation*}
      \begin{aligned}
      &\maximize_{\mu} \sum_{i=1}^n \log P(x_i \mid \mu)\\
      \Rightarrow &\maximize_{\mu} \sum_{i=1}^n (\log \frac{1}{\sqrt{2 \pi \delta^2}} - \frac{(x - \mu)^2}{2 \delta^2})\\
      \Rightarrow &\minimize_{\mu} \sum_{i=1}^n (x_i - \mu)^2
      \end{aligned}
    \end{equation*}
    Here we define $f(\mu) = \sum_{i=1}^n (x_i - \mu)^2$ to find the $\mu^*$. And $f(\mu)$ is a convex function. There exists $\frac{\partial f}{\partial \mu} = 2\sum_i^n (\mu - x_i) = 0$ when $\mu$=$\mu^*$ .\\
    Then $\mu^* = \frac{\sum_{i=1}^n x_i}{n}$.
    %  \defeq \max\{1 - yz, 0\}$.

  \item
    The MAP problem can be writen as:
    \begin{equation*}
      \maximize_{\mu} P(\mu \mid x_1,...,x_n)
    \end{equation*}
    which is equivalent to:
    \begin{equation*}
      \begin{aligned}
      &\maximize_{\mu} P(\mu\mid x_1,...,x_n)\\
      \Rightarrow &\maximize_{\mu} P(x_1,...,x_n\mid\mu)P(\mu)\\
      \Rightarrow &\maximize_{\mu} \prod_{i=1}^n P(x_i\mid\mu)P(\mu)\\
      \Rightarrow &\maximize_{\mu} \sum_{i=1}^n \log P(x_i\mid\mu) + \log P(\mu)\\
      \Rightarrow &\maximize_{\mu} -\sum_{i=1}^n \frac{(x_i-\mu)^2}{2\delta^2} - \log \sqrt{2\pi\theta^2} - \frac{(\mu - \nu)^2}{2 \delta^2}\\
      \Rightarrow &\minimize_{\mu} \sum_{i=1}^n \frac{(x_i-\mu)^2}{2\delta^2} + \frac{(\mu - \nu)^2}{2 \theta^2}\\
      \end{aligned}
    \end{equation*}
    Here we define $g(\mu) = \sum_{i=1}^n \frac{(x_i-\mu)^2}{2\delta^2} + \frac{(\mu - \nu)^2}{2 \theta^2}$ to find the $\mu^*$. There exists $\frac{\partial g}{\partial \mu} = \sum_{i=1}^n \frac{\mu-x_i}{\delta^2} + \frac{\mu - \nu}{\theta^2} = 0$ when $\mu$=$\mu^*$ .\\
    Then $\mu^* = \frac{\theta^2\sum_{i=1}^n x_i + \delta^2\nu}{\theta^2n+\delta^2}$.\\
    When $n \rightarrow \infty$, $\mu^* = \frac{\sum_{i=1}^n x_i}{n}$ and MLE and MAP is equal.
  \end{enumerate}

% The solution of 1.3
\item \small{\textsl{Here we define for matrix $\boldsymbol{X}$, $\boldsymbol{X}_i$ is the $i$th column vector for $\boldsymbol{X}$, and $\boldsymbol{X}_{ij}$ is the element in the $i$th row and $j$th column. Other symbols' definations are the same as those in the problem.}}\\
First, the square error can be written as 
  \begin{equation*}
    \begin{aligned}
      J(\boldsymbol \Theta) 
      &= \frac{1}{2} \sum_{i = 1}^m \sum_{j = 1}^l (({\boldsymbol \Theta}^T \boldsymbol{x}^{(i)})_j - \boldsymbol{y}_j^{(i)})^2\\
      &= \frac{1}{2} \sum_{i = 1}^m \sum_{j = 1}^l (\sum_{k=1}^n \boldsymbol{\Theta}_{kj} \boldsymbol{x}^{(i)}_k - \boldsymbol{y}_j^{(i)})^2
    \end{aligned}
  \end{equation*}
  In order to compute the solution, it needs to find the minimum $J(\Theta)$ where $\frac{\partial J}{\partial \Theta}=0$. To find the solution, the derivative is:
  \begin{equation*}
    \begin{aligned}
      \frac{\partial J}{\partial \boldsymbol{\Theta}_{\alpha \beta}}
      &= \sum_{i = 1}^m (\boldsymbol{x}_\alpha^{(i)}(\sum_k^n \boldsymbol{\Theta}_{k \beta} \boldsymbol{x}_k^{(i)}-\boldsymbol{y}_{\beta}^{(i)}))\\
      &= \sum_{i = 1}^m (\boldsymbol{x}_\alpha^{(i)}(\boldsymbol{\Theta}_\beta^T \boldsymbol{x}^{(i)}-\boldsymbol{y}_{\beta}^{(i)}))\\
    \end{aligned}
  \end{equation*}
  
  Then we have $\frac{\partial J}{\partial \Theta}=0$, which means:
  \begin{equation*}
    \begin{aligned}
      \sum_{i = 1}^m \boldsymbol{x}_\alpha^{(i)}(\boldsymbol{\Theta}_\beta^T \boldsymbol{x}^{(i)})
      &= \sum_{i = 1}^m \boldsymbol{x}^{(i)}_\alpha \boldsymbol{y}_{\beta}^{(i)})\\
      \Rightarrow \boldsymbol{X}_\alpha^T \boldsymbol{X} \boldsymbol{\Theta}_\beta &= \boldsymbol{X}_\alpha^T \boldsymbol{Y}_{\beta}\\
      \Rightarrow \boldsymbol{X}^T \boldsymbol{X} \boldsymbol{\Theta}_\beta &= \boldsymbol{X}^T \boldsymbol{Y}_{\beta}\\
    \end{aligned}
  \end{equation*}
  So, for $\beta \in (1,2,...,l)$ we have
  \begin{equation*}
    \boldsymbol{\Theta}_\beta = (\boldsymbol{X}^T \boldsymbol{X})^{-1} \boldsymbol{X}^T \boldsymbol{Y}_{\beta}
  \end{equation*}
  Finally,
  \begin{equation*}
    \boldsymbol{\Theta} = (\boldsymbol{X}^T \boldsymbol{X})^{-1} \boldsymbol{X}^T \boldsymbol{Y}
  \end{equation*}


% The solution of 1.4
\item Based on $\Sigma$ is symmetrical as well as the properties of matrix trace, the multivariate normal distribution can be written as
\begin{equation*}
  \begin{aligned}
    P_{\boldsymbol y}(\boldsymbol y;\boldsymbol \mu,\boldsymbol \Sigma)
    &= \frac{1}{(2\pi)^{\frac{n}{2}}|\boldsymbol \Sigma|^{\frac{1}{2}}} \exp(-\frac{1}{2}(\boldsymbol{y}-\boldsymbol \mu)^T {\boldsymbol \Sigma}^{-1}(\boldsymbol y-\boldsymbol \mu))\\
    &= (2\pi)^{-\frac{n}{2}}|\boldsymbol \Sigma|^{-\frac{1}{2}} \exp(-\frac{1}{2}(\boldsymbol{y}^T{\boldsymbol \Sigma}^{-1}-\boldsymbol{\mu}^T{\boldsymbol \Sigma}^{-1}) (\boldsymbol{y}-\boldsymbol{\mu}))\\
    &= (2\pi)^{-\frac{n}{2}}|\boldsymbol \Sigma|^{-\frac{1}{2}} \exp(-\frac{1}{2}(\boldsymbol{y}^T{\boldsymbol \Sigma}^{-1} \boldsymbol{y}-\boldsymbol{\mu}^T{\boldsymbol \Sigma}^{-1}\boldsymbol{y} - \boldsymbol{y}^T{\boldsymbol \Sigma}^{-1} \boldsymbol{\mu} + \boldsymbol{\mu}^T{\boldsymbol \Sigma}^{-1}\boldsymbol{\mu})\\
    &= (2\pi)^{-\frac{n}{2}}|\boldsymbol \Sigma|^{-\frac{1}{2}} \exp(-\frac{1}{2} tr(\boldsymbol{y}^T{\boldsymbol \Sigma}^{-1} \boldsymbol{y}-\boldsymbol{\mu}^T{\boldsymbol \Sigma}^{-1}\boldsymbol{y} - \boldsymbol{y}^T{\boldsymbol \Sigma}^{-1} \boldsymbol{\mu} + \boldsymbol{\mu}^T{\boldsymbol \Sigma}^{-1}\boldsymbol{\mu})\\
    &= (2\pi)^{-\frac{n}{2}}|\boldsymbol \Sigma|^{-\frac{1}{2}} \exp(-\frac{1}{2} (tr(\boldsymbol{y}^T{\boldsymbol \Sigma}^{-1} \boldsymbol{y})-tr(\boldsymbol{\mu}^T{\boldsymbol \Sigma}^{-1}\boldsymbol{y}) - tr(\boldsymbol{y}^T{\boldsymbol \Sigma}^{-1} \boldsymbol{\mu}) + tr(\boldsymbol{\mu}^T{\boldsymbol \Sigma}^{-1}\boldsymbol{\mu}))\\
    &= (2\pi)^{-\frac{n}{2}}|\boldsymbol \Sigma|^{-\frac{1}{2}} \exp(tr(\boldsymbol{\mu}^T{\boldsymbol \Sigma}^{-1}\boldsymbol{y}) -\frac{1}{2}tr({\boldsymbol \Sigma}^{-1} \boldsymbol{y} \boldsymbol{y}^T)-\frac{1}{2} \boldsymbol{\mu}^T{\boldsymbol \Sigma}^{-1}\boldsymbol{\mu})\\
  \end{aligned}
\end{equation*}
Thus, we can see that multivariate normal distribution is an exponential family with:
\begin{equation*}
  \begin{aligned}
    \eta &= 
    \begin{pmatrix}
      {\boldsymbol \Sigma}^{-1}\boldsymbol{\mu}\\
      -\frac{1}{2}{\boldsymbol \Sigma}^{-1}
    \end{pmatrix}\\
    b(\boldsymbol y) &= (2\pi)^{-\frac{n}{2}}|\boldsymbol \Sigma|^{-\frac{1}{2}}\\
    T(\boldsymbol y)&=
    \begin{pmatrix}
      \boldsymbol{y}\\
      \boldsymbol{y y}^T
    \end{pmatrix}\\
    a(\eta) &= \frac{1}{2} \boldsymbol{\mu}^T{\boldsymbol \Sigma}^{-1}\boldsymbol{\mu}
  \end{aligned}
\end{equation*}
\end{enumerate}
  
  % \newpage
  
  % \appendix
  % \section{Source code}
  % \label{sec:a:code}
  % % \lstlistoflistings
  % Source code for plotting Figure \ref{fig:1} is shown as follows.
  % \lstinputlisting{matlabscript.m}
  
\end{document}
%%% Local Variables:
%%% mode: latex
%%% TeX-master: t
%%% End:
