% Homework template for Learning from Data
% by Xiangxiang Xu <xiangxiangxu.thu@gmail.com>
% LAST UPDATE: October 8, 2018
\documentclass[a4paper]{article}
\usepackage[T1]{fontenc}
\usepackage{amsmath, amssymb, amsthm}
% amsmath: equation*, amssymb: mathbb, amsthm: proof
\usepackage{moreenum}
\usepackage{mathtools}
\usepackage{url}
\usepackage{enumitem}
\usepackage{bm}
\usepackage{graphicx}
\usepackage{subcaption}
\usepackage{booktabs} % toprule
\usepackage[mathcal]{eucal}
\usepackage{dsfont}
\usepackage[numbered,framed]{matlab-prettifier}
\input{lddef}

\lstset{
  style              = Matlab-editor,
  captionpos         =b,
  basicstyle         = \mlttfamily,
  escapechar         = ",
  mlshowsectionrules = true,
}
\begin{document}
\courseheader



\newcounter{hwcnt}
\setcounter{hwcnt}{2} % set to the times of Homework

\begin{center}
  \underline{\bf Homework 3}%\thehwcnt} \\
\end{center}
\begin{flushleft}
  TIAN Chenyu\hfill
  \today
\end{flushleft}
\hrule

\vspace{2em}
\setlist[enumerate,1]{label=\thehwcnt.\arabic*.}
\setlist[enumerate,2]{label=(\alph*)}
\setlist[enumerate,3]{label=\roman*.}
\setlist[enumerate,4]{label=\greek*)}

\flushleft
\rule{\textwidth}{1pt}
\begin{itemize}
\item {\bf Acknowledgments: \/} 
  This template takes some materials from course CSE 547/Stat 548 of Washington University: \small{\url{https://courses.cs.washington.edu/courses/cse547/17sp/index.html}}.
\item {\bf Collaborators: \/}
  I finish this homework by myself.
  % \begin{itemize}
  % \item 1.2 (b) was solved with the help from \underline{\hspace{3em}}.
  % \item Discussion with \underline{\hspace{3em}} helped me finishing 1.3.
  % \end{itemize}
\end{itemize}
\rule{\textwidth}{1pt}

\vspace{2em}

% You may use \texttt{enumerate} to generate answers for each question:

\begin{enumerate}
  \setlength{\itemsep}{3\parskip}
\item The log-likelihood of the naive Bayes model can be writen as  
\begin{equation*}
    % \tag{P}
    \begin{aligned}
      l(\phi_y,\phi_j (x \mid y)) 
      &= \sum_{i = 1}^m \log p(x^{(i)}, y^{(i)})\\
      &= \sum_{i = 1}^m (\log p(x^{(i)} \mid y^{(i)}) + \log p(y^{(i)}))\\
      &= \sum_{i = 1}^m (\sum_{j = 1}^d \log p(x^{(i)}_j \mid y^{(i)}) + \log p(y^{(i)}))\\
      &= \sum_{i = 1}^m \sum_{j = 1}^d \log \phi_j (x^{(i)}_j \mid y^{(i)}) + \sum_{i = 1}^m \log \phi_{y^{(i)}}\\
    \end{aligned}
    \label{eq:svm:p}
  \end{equation*}

  Also, there exist constrains for $\phi_y,\phi_j (x \mid y)$:
  \begin{equation*}
    \begin{aligned}
      \sum_{y=1}^k \phi_y &= 1\\
      \sum_{x \in \{0,1\}} \phi_j (x \mid y) &= 1 \quad y \in \{1,...,K\}\\
    \end{aligned}
  \end{equation*}
Then to find the maximum value of $l$ with bringing these constrains into equation using lagrange multiplier
\begin{equation*}
  \begin{aligned}
    g(l, \lambda, \lambda_{jk}) = \sum_{i = 1}^m \sum_{j = 1}^d \log \phi_j (x^{(i)}_j \mid y^{(i)}) + \sum_{i = 1}^m \log \phi_{y^{(i)}} \\
      + \lambda(\sum_{y=1}^k\phi_y - 1) +\lambda_{jk}(\sum_{x \in \{0,1\}}\phi_j(x \mid k)-1)
  \end{aligned}
\end{equation*}

Using lagrange multiplier method to find the parameters of the maximum value, there exists:
\begin{equation*}
  \begin{gathered}
    \frac{dg}{d \phi_y} = \sum_{i = 1}^m \1 (y^{(i)}=y) \frac{1}{\phi_y} + \lambda = 0\\
    \Rightarrow \sum_{i = 1}^m \1 (y^{(i)}=1) \frac{1}{\phi_1}=\sum_{i = 1}^m \1 (y^{(i)}=2) \frac{1}{\phi_2}=...=\sum_{i = 1}^m \1 (y^{(i)}=K) \frac{1}{\phi_K}
  \end{gathered}
\end{equation*}
Then here has
\begin{equation*}
  \phi_y = \frac{\sum_{i = 1}^m \1 (y^{(i)}=y)}{m}
\end{equation*}

Also, 
\begin{equation*}
  \begin{gathered}
    \frac{dg}{d \phi_j (x \mid y)} = \sum_{i = 1}^m \1 (y^{(i)}=y) \1 (x^{(i)}_j=x) \frac{1}{\phi_j (x \mid y)} + \lambda_j (x \mid y) = 0\\
    \Rightarrow \sum_{i = 1}^m \1 (y^{(i)}=y) \1 (x^{(i)}_j=0) \frac{1}{\phi_j (x=0 \mid y)} = \sum_{i = 1}^m \1 (y^{(i)}=y) \1 (x^{(i)}_j=1) \frac{1}{\phi_j (x=1 \mid y)}\\
  \end{gathered}
\end{equation*}
Thus,
\begin{equation*}
  \phi_j (x \mid y) = \frac{\sum_{i = 1}^m \1 (y^{(i)}=y) \1 (x^{(i)}_j=x)}{\sum_{i = 1}^m \1 (y^{(i)}=y)}
\end{equation*}
\end{enumerate}
  
  % \newpage
  
  % \appendix
  % \section{Source code}
  % \label{sec:a:code}
  % % \lstlistoflistings
  % Source code for plotting Figure \ref{fig:1} is shown as follows.
  % \lstinputlisting{matlabscript.m}
  
\end{document}
%%% Local Variables:
%%% mode: latex
%%% TeX-master: t
%%% End:
