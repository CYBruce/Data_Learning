% Homework template for Learning from Data
% by Xiangxiang Xu <xiangxiangxu.thu@gmail.com>
% LAST UPDATE: October 8, 2018
\documentclass[a4paper]{article}
\usepackage[T1]{fontenc}
\usepackage{amsmath, amssymb, amsthm}
% amsmath: equation*, amssymb: mathbb, amsthm: proof
\usepackage{moreenum}
\usepackage{mathtools}
\usepackage{url}
\usepackage{enumitem}
\usepackage{bm}
\usepackage{graphicx}
\usepackage{subcaption}
\usepackage{booktabs} % toprule
\usepackage[mathcal]{eucal}
\usepackage{dsfont}
\usepackage[numbered,framed]{matlab-prettifier}
%% Definitions for Learning from Data
%% UPDATE: October 8, 2018 by Xiangxiang 
\newcommand{\theterm}{Fall 2019}

\newcommand{\thecoursename}{
Tsinghua-Berkeley Shenzhen Institute\\
%\vspace*{0.1in}
\textsc{Learning from Data}
}

\newcommand{\courseheader}{
\vspace*{-1in}
\begin{center}
\thecoursename \\
\theterm
\vspace*{0.1in}
\hrule
\end{center}
}
\newcommand{\uc}{\underline{c}}    % c, vec
\newcommand{\uv}{\underline{v}}    % x, vec
\newcommand{\uw}{\underline{w}}    % w, vec
\newcommand{\ux}{\underline{x}}    % x, vec
\newcommand{\uy}{\underline{y}}    % y, vec
\newcommand{\uz}{\underline{z}}    % z, vec
\newcommand{\um}{\underline{m}}    % m, vec
\newcommand{\ut}{\underline{t}}    % t, vec
\newcommand{\bc}{\bm{c}}    % c, vec
\newcommand{\bv}{\bm{v}}    % x, vec
\newcommand{\bw}{\bm{w}}    % w, vec
\newcommand{\bx}{\bm{x}}    % x, vec
\newcommand{\by}{\bm{y}}    % y, vec
\newcommand{\bz}{\bm{z}}    % z, vec
% \newcommand{\bm}{\bm{m}}    % m, vec
\newcommand{\bt}{\bm{t}}    % t, vec

\newcommand{\balpha}{\bm{\alpha}}    % alpha, vec
\newcommand{\bxi}{\bm{\xi}}    % xi, vec


\newcommand{\rvx}{\mathsf{x}}    % x, r.v.
\newcommand{\rvy}{\mathsf{y}}    % y, r.v.
\newcommand{\rvz}{\mathsf{z}}    % z, r.v.
\newcommand{\rvw}{\mathsf{w}}    % w, r.v.
\newcommand{\rvv}{\mathsf{v}}    % v, r.v.
\newcommand{\rvm}{\mathsf{m}}    % m, r.v.
\newcommand{\rvt}{\mathsf{t}}    % t, r.v.
\newcommand{\rvH}{\mathsf{H}}    % H, r.v.
\newcommand{\urvx}{\underline{\mathsf{x}}}    % x, r.v. vec
\newcommand{\urvy}{\underline{\mathsf{y}}}    % y, r.v. vec
\newcommand{\urvz}{\underline{\mathsf{z}}}    % z, r.v. vec
\newcommand{\urvw}{\underline{\mathsf{w}}}    % w, r.v. vec
\newcommand{\urvt}{\underline{\mathsf{t}}}    % t, r.v. vec
\newcommand{\defeq}{\triangleq} %\coloneqq
\newcommand{\reals}{\mathbb{R}}
\newcommand{\T}{\mathrm{T}}    % transpose
\newcommand{\BLS}{\mathrm{BLS}}    % BLS
\newcommand{\LLS}{\mathrm{LLS}}    % LLS
\newcommand{\MVU}{\mathrm{MVU}}    % MVU

\DeclareMathOperator*{\maximize}{maximize}    % maximize
\DeclareMathOperator*{\minimize}{minimize}    % minimize
\newcommand{\st}{\mathrm{subject~to}}    % minimize


% \newcommand{\E}[1]{\mathbb{E}\left[{#1}\right]}
% \newcommand{\Prob}[1]{\mathbb{P}\left({#1}\right)}
\DeclareMathOperator*{\argmax}{arg\,max}
\DeclareMathOperator*{\argmin}{arg\,min}
\DeclareMathOperator*{\argsup}{arg\,sup}
\DeclareMathOperator*{\arginf}{arg\,inf}
\DeclareMathOperator{\Var}{Var}
\DeclareMathOperator{\Cov}{Cov}
\DeclareMathOperator{\MSE}{MSE}
\DeclareMathOperator{\1}{\mathds{1}}
\DeclareMathOperator{\E}{\mathbb{E}}
\DeclareMathOperator{\Prob}{\mathbb{P}}

\newcommand\independent{\protect\mathpalette{\protect\independenT}{\perp}}
\def\independenT#1#2{\mathrel{\rlap{$#1#2$}\mkern2mu{#1#2}}}


\lstset{
  style              = Matlab-editor,
  captionpos         =b,
  basicstyle         = \mlttfamily,
  escapechar         = ",
  mlshowsectionrules = true,
}
\begin{document}
\courseheader



\newcounter{hwcnt}
\setcounter{hwcnt}{2} % set to the times of Homework

\begin{center}
  \underline{\bf Writing Homework 2}%\thehwcnt} \\
\end{center}
\begin{flushleft}
  TIAN Chenyu\hfill
  \today
\end{flushleft}
\hrule

\vspace{2em}
\setlist[enumerate,1]{label=\thehwcnt.\arabic*.}
\setlist[enumerate,2]{label=(\alph*)}
\setlist[enumerate,3]{label=\roman*.}
\setlist[enumerate,4]{label=\greek*)}

\flushleft
\rule{\textwidth}{1pt}
\begin{itemize}
\item {\bf Acknowledgments: \/} 
  This template takes some materials from course CSE 547/Stat 548 of Washington University: \small{\url{https://courses.cs.washington.edu/courses/cse547/17sp/index.html}}.
\item {\bf Collaborators: \/}
  I finish this homework by myself.
  % \begin{itemize}
  % \item 1.2 (b) was solved with the help from \underline{\hspace{3em}}.
  % \item Discussion with \underline{\hspace{3em}} helped me finishing 1.3.
  % \end{itemize}
\end{itemize}
\rule{\textwidth}{1pt}

\vspace{2em}

% You may use \texttt{enumerate} to generate answers for each question:

\begin{enumerate}
  \setlength{\itemsep}{3\parskip}
\item Define $\boldsymbol{P} = \boldsymbol{X}\left(\boldsymbol{X}^{\mathrm{T}} \boldsymbol{X}\right)^{-1} \boldsymbol{X}^{\mathrm{T}}$, for a given vector $\boldsymbol{v}$
\begin{equation*}
  \boldsymbol{v} = \boldsymbol{P} \boldsymbol{v}+ (\boldsymbol{v}-\boldsymbol{P} \boldsymbol{v})
\end{equation*}
If we can prove that $\boldsymbol{v}-\boldsymbol{P} \boldsymbol{v}$ is orthogonal to both $\boldsymbol{P} \boldsymbol{v}$ and the column space of $\boldsymbol{X}$,
we can prove that matrix $\boldsymbol{P}$ project $\boldsymbol{v}$ onto column space of $\boldsymbol{X}$.\\
So this problem is equvalent to prove:
  \begin{equation*}
    % \tag{P}
    \begin{aligned}
      (\boldsymbol{P} \boldsymbol{v})^T (\boldsymbol{v}-\boldsymbol{P} \boldsymbol{v}) = 0\\
      \boldsymbol{X}^T (\boldsymbol{v}-\boldsymbol{P} \boldsymbol{v}) = 0\\
    \end{aligned}
    \label{eq:svm:p}
  \end{equation*}

  Proof:
  \begin{equation*}
    \begin{aligned}
      (\boldsymbol{P} \boldsymbol{v})^T (\boldsymbol{v}-\boldsymbol{P} \boldsymbol{v})
      &= (\boldsymbol{X}\left(\boldsymbol{X}^{\mathrm{T}} \boldsymbol{X}\right)^{-1} \boldsymbol{X}^{\mathrm{T}}\boldsymbol{v})^T \boldsymbol{v}\\
      &\quad - (\boldsymbol{X}\left(\boldsymbol{X}^{\mathrm{T}} \boldsymbol{X}\right)^{-1} \boldsymbol{X}^{\mathrm{T}}\boldsymbol{v})^T(\boldsymbol{X}\left(\boldsymbol{X}^{\mathrm{T}} \boldsymbol{X}\right)^{-1} \boldsymbol{X}^{\mathrm{T}}\boldsymbol{v})\\
      &=\boldsymbol{v}^T \boldsymbol{X} ((\boldsymbol{X}^T \boldsymbol{X})^{-1})^T \boldsymbol{X}^T \boldsymbol{v} \\
      &\quad -\boldsymbol{v}^T \boldsymbol{X} (\boldsymbol{X}^T \boldsymbol{X})^{-1} \boldsymbol{X}^T \boldsymbol{X} (\boldsymbol{X}^T \boldsymbol{X})^{-1} \boldsymbol{X}^T \boldsymbol{v}\\
      &=\boldsymbol{v}^T (\boldsymbol{X}^T \boldsymbol{X})^{-1} \boldsymbol{X}\left(\boldsymbol{X}^{\mathrm{T}} \boldsymbol{X}\right)^{-1} \boldsymbol{X}^{\mathrm{T}}\boldsymbol{v}\\ 
      &\quad -\boldsymbol{v}^T \boldsymbol{X}\left(\boldsymbol{X}^{\mathrm{T}} \boldsymbol{X}\right)^{-1} \boldsymbol{X}^{\mathrm{T}}\boldsymbol{v}\\
      &=0
    \end{aligned}
  \end{equation*}
  \begin{equation*}
    \begin{aligned}
      \boldsymbol{X}^T (\boldsymbol{v}-\boldsymbol{P} \boldsymbol{v})
      &= \boldsymbol{X}^{\mathrm{T}} \boldsymbol{v} - \boldsymbol{X}^{\mathrm{T}} \boldsymbol{X}^T \boldsymbol{P} \boldsymbol{v}\\
      &=\boldsymbol{X}^{\mathrm{T}} \boldsymbol{v} - \boldsymbol{X}^T \boldsymbol{X} (\boldsymbol{X}^T \boldsymbol{X})^{-1} \boldsymbol{X}^T \boldsymbol{v}\\
      &=\boldsymbol{X}^{\mathrm{T}} \boldsymbol{v} - \boldsymbol{X}^{\mathrm{T}} \boldsymbol{v}\\
      &=0
    \end{aligned}
  \end{equation*}
Thus, $\boldsymbol{P} = \boldsymbol{X}\left(\boldsymbol{X}^{\mathrm{T}} \boldsymbol{X}\right)^{-1} \boldsymbol{X}^{\mathrm{T}}$ project $\boldsymbol{v}$ onto column space of $\boldsymbol{X}$.

Then need to prove that $\boldsymbol{\theta}$ correspond to an orthogonal projection of the vector $\boldsymbol{y}$ onto the column space of $\boldsymbol{X}$.
%  ???????

\item Then to find the maximum value of $l$ with bringing these constrains into equation using lagrange multiplier
\begin{equation*}
  \begin{aligned}
    g(l, \lambda, \lambda_{jk}) = \sum_{i = 1}^m \sum_{j = 1}^d \log \phi_j (x^{(i)}_j \mid y^{(i)}) + \sum_{i = 1}^m \log \phi_{y^{(i)}} \\
      + \lambda(\sum_{y=1}^k\phi_y - 1) +\lambda_{jk}(\sum_{x \in \{0,1\}}\phi_j(x \mid k)-1)
  \end{aligned}
\end{equation*}
\item Questions about SVM
\begin{enumerate}
  \item half done
  \item From the KKT condition, here exists:
    \begin{equation*}
      \begin{gathered}
        \sum_{i=1}^{l} \alpha_{i}^{\star}\left[y_{i}\left(\boldsymbol{w}^{\star \mathrm{T}} \boldsymbol{x}_{i}+b^{\star}\right)-1\right]=0\\
        \Rightarrow \quad \sum_{i=1}^{l} \alpha_{i}^{\star} y_{i} \boldsymbol{w}^{\star \mathrm{T}} \boldsymbol{x}_{i}+ \sum_{i=1}^{l} \alpha_{i}^{\star} y_{i} b^{\star} = \sum_{i=1}^{l} \alpha_{i}^{\star}\\
        \Rightarrow \quad \sum_{i=1}^{l} \alpha_{i}^{\star} y_{i} \boldsymbol{w}^{\star \mathrm{T}} \boldsymbol{x}_{i}+ b^{\star} \sum_{i=1}^{l} \alpha_{i}^{\star} y_{i} = \sum_{i=1}^{l} \alpha_{i}^{\star}\\
        \Rightarrow \quad \sum_{i=1}^{l} \alpha_{i}^{\star} y_{i} \boldsymbol{w}^{\star \mathrm{T}} \boldsymbol{x}_{i} = \sum_{i=1}^{l} \alpha_{i}^{\star}\\
        \Rightarrow \quad \sum_{i=1}^{l} \sum_{j=1}^{l} \alpha_{i}^{\star} \alpha_{j}^{\star} y_{i} y_{j}\left\langle\boldsymbol{x}_{i}, \boldsymbol{x}_{j}\right\rangle = \sum_{i=1}^{l} \alpha_{i}^{\star}\\
      \end{gathered}
    \end{equation*}
    Then, using the equation above, it has
    \begin{equation*}
      \begin{aligned}
        \frac{1}{2}\left\|\boldsymbol{w}^{\star}\right\|_{2}^{2}
        &=\sum_{i=1}^{l} \alpha_{i}^{\star}-\frac{1}{2} \sum_{i=1}^{l} \sum_{j=1}^{l} \alpha_{i}^{\star} \alpha_{j}^{\star} y_{i} y_{j}\left\langle\boldsymbol{x}_{i}, \boldsymbol{x}_{j}\right\rangle\\
        &=\frac{1}{2} \sum_{i=1}^{l} \alpha_{i}^{\star}\\
      \end{aligned}
    \end{equation*}
\end{enumerate}

\item 
\begin{enumerate}
  \item a
  \item To prove a convex function
  \begin{equation*}
    \begin{aligned}
      f(\boldsymbol{\omega} ,b) = \frac{1}{2}\|\boldsymbol{w}\|_{2}^{2}+C \sum_{i=1}^{l} \ell\left(y_{i}, \boldsymbol{w}^{\mathrm{T}} \boldsymbol{x}_{i}+b\right)
    \end{aligned}
  \end{equation*}
  proof:
  \begin{equation*}
    \begin{aligned}
      &\|\theta \boldsymbol{w}_{1}+(1-\theta) \boldsymbol{w}_{2}\|_{2}^{2} -\theta \| \boldsymbol{w}_{1}\|_{2}^{2}-(1-\theta) \|\boldsymbol{w}_{2}\|_{2}^{2}\\
      =& \theta^{2}\|\boldsymbol{w}_{1}\|_{2}^{2} + 2\theta(1-\theta)\boldsymbol{w}_{1}^{T} \boldsymbol{w}_{2}+ (1-\theta)^{2} \|\boldsymbol{w}_{2}\|_{2}^{2} -\theta \| \boldsymbol{w}_{1}\|_{2}^{2}-(1-\theta) \|\boldsymbol{w}_{2}\|_{2}^{2}\\
      =& 2\theta(1-\theta)\boldsymbol{w}_{1}^{T} \boldsymbol{w}_{2} - \theta(1-\theta)\|\boldsymbol{w}_{1}\|_{2}^{2} -\theta(1-\theta)\|\boldsymbol{w}_{2}\|_{2}^{2}\\
      \leq &2\theta(1-\theta)\boldsymbol{w}_{1}^{T} \boldsymbol{w}_{2} - \theta\|\boldsymbol{w}_{1}\|_{2}^{2} -\theta\|\boldsymbol{w}_{2}\|_{2}^{2}\\
      \leq &-\theta \|\boldsymbol{w}_{1} - \boldsymbol{w}_{2}\|_{2}^{2}\\
      \leq &0\\
      \Rightarrow \quad& \|\theta \boldsymbol{w}_{1}+(1-\theta) \boldsymbol{w}_{2}\|_{2}^{2}\leq \theta \| \boldsymbol{w}_{1}\|_{2}^{2} + (1-\theta) \| \boldsymbol{w}_{2}\|_{2}^{2}
    \end{aligned}
  \end{equation*}
  So $\|\boldsymbol{w}\|_{2}^{2}$ is a convex function.
  \begin{equation*}
    \begin{aligned}
      &\ell \left(y_i, (\theta \boldsymbol{w}_{1}+(1-\theta) \boldsymbol{w}_{2})^{T} \boldsymbol{x_i}+\theta b_{1}+(1-\theta) b_{2}\right)\\
      =& \max \{1-y_i \left((\theta \boldsymbol{w}_{1}+(1-\theta) \boldsymbol{w}_{2})^{T} \boldsymbol{x_i}+\theta b_{1}+(1-\theta) b_{2}\right), 0\}\\
      \leq &\max \{\theta-y_i (\theta \boldsymbol{w}_{1}^{T}\boldsymbol{x_i}+\theta b_{1}) + (1-\theta)-y_i ((1-\theta) \boldsymbol{w}_{2}^{T}\boldsymbol{x_i} +(1-\theta) b_{2}), 0\}\\
      \leq &\max \{\theta-y_i (\theta \boldsymbol{w}_{1}^{T}\boldsymbol{x_i}+\theta b_{1}) ), 0\} + \max \{(1-\theta)-y_i ((1-\theta) \boldsymbol{w}_{2}^{T}\boldsymbol{x_i} +(1-\theta) b_{2}), 0\}\\
      \leq &\theta \max \{1-y_i (\boldsymbol{w}_{1}^{T}\boldsymbol{x_i}+b_{1}) ), 0\} + (1-\theta)\max \{1-y_i \boldsymbol{w}_{2}^{T}\boldsymbol{x_i} +b_{2}), 0\}\\ 
      \leq &\theta \ell \left(y_i, \boldsymbol{w}_{1}^{T} \boldsymbol{x}_{i}+b_{1}\right)+(1-\theta) \ell \left(y_i, \boldsymbol{w}_{2}^{T} \boldsymbol{x}_{i}+b_{2}\right)
    \end{aligned}
  \end{equation*}
  So $\ell(\boldsymbol{w}^{\mathrm{T}} \boldsymbol{x}_{i}+b)$ is a convex function.\\
  The non-negative weighted sum of convex functions is still a convex function. And $C \geq 0$. \\
  Thus the objective function $f(\boldsymbol{\omega} ,b) = \frac{1}{2}\|\boldsymbol{w}\|_{2}^{2}+C \sum_{i=1}^{l} \ell\left(y_{i}, \boldsymbol{w}^{\mathrm{T}} \boldsymbol{x}_{i}+b\right)$ is convex.
\end{enumerate}

\end{enumerate}
  
  % \newpage
  
  % \appendix
  % \section{Source code}
  % \label{sec:a:code}
  % % \lstlistoflistings
  % Source code for plotting Figure \ref{fig:1} is shown as follows.
  % \lstinputlisting{matlabscript.m}
  
\end{document}
%%% Local Variables:
%%% mode: latex
%%% TeX-master: t
%%% End:
